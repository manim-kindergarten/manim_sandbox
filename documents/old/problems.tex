\documentclass[UTF8]{ctexart}
\usepackage{amsmath}
\usepackage{amssymb}
\usepackage{geometry}
\usepackage{listings}
\usepackage{graphicx}
\usepackage[colorlinks,linkcolor=blue]{hyperref}
\geometry{a4paper, scale=0.80}

\lstset{
	basicstyle=\fontspec{Consolas}
}

\title{manim常见问题 v1.1}
\author{鹤翔万里}

\begin{document}

\maketitle
\tableofcontents
\newpage

\section{安装问题}

安装时最好不要看\texttt{README.md}自己研究。推荐一视数学卷毛杨的两个教程\url{https://www.bilibili.com/video/av38126904}
\qquad\url{https://www.bilibili.com/read/cv4139851}

\subsection{python安装}

\subsubsection*{Q1:使用anaconda,命令行输入python无反应或报错}
考虑path环境变量是否填全(安装anaconda时是否勾选添加到path变量)path变量里应该有
\begin{align*}
&\mathtt{<your\_path>\backslash Anaconda3}\\
&\mathtt{<your\_path>\backslash Anaconda3\backslash Scripts}\\
&\mathtt{<your\_path>\backslash Anaconda3\backslash Library\backslash bin}
\end{align*}

\subsubsection*{Q2: pip install ...时满屏红字报错,或者安装过慢}
更换国内镜像源,使用\texttt{pip install -r requirements.txt -i https://pypi.tuna.tsinghua.\\edu.cn/simple}
代替\texttt{pip install -r requirements.txt}临时换源

\subsubsection*{Q3: pip安装pycairo总是失败}
下载pycairo的whl包\texttt{pycairo-1.18.2-cp37-cp37m-win\_amd64.whl}(群文件中有),并手动安装\texttt{pip install pycairo.....whl}

\section{运行时问题}

\subsection{某一版manim的bug}
\subsubsection*{Q1: 运行OpeningManimExample报错\texttt{TypeError: descriptor '\_\_init\_\_' of 'super' object needs an argument}}
这是Grant对manimlib进行升级时不小心搞出的bug,但是最新版已经改过来了

如果\texttt{manimlib/mobject/coordinate\_systems.py}第287行为\texttt{super.\_\_init\_\_(**kwargs)},请改为
\texttt{super().\_\_init\_\_(**kwargs)}或\texttt{Axes.\_\_init\_\_(self, **kwargs)}

\subsection{\texttt{import}问题}
\subsubsection*{Q1: 没有模块\texttt{big\_ol\_pile\_of\_manim\_imports}}
将文件中的\texttt{from big\_ol\_pile\_of\_manim\_imports import *}改成
\texttt{from manimlib.imports import *}

\subsection{\LaTeX 问题}

\subsubsection*{Q1: 报错Latex error converting to dvi}
先不要管错误在哪,先把manimlib/constants.py中的\texttt{TEX\_USE\_CTEX}改成\texttt{True}再运行

\subsubsection*{Q2: 报错xelatex error converting to xdv}
若为Windows系统,先把manimlib/constants.py的第29行\texttt{MEDIA\_DIR = "./media"}改成
$$
\texttt{MEDIA\_DIR = os.path.join(os.getcwd(), "media")}
$$
再进行尝试\\

\textbf{I.若安装的\TeX 发行版为MiK\TeX}
考虑 1.MiK\TeX 的有关路径是否添加到环境变量中 2.是否有包没有装全

对于2. 可以正常运行一遍WriteStuff场景,看是否有框弹出提示install什么东西,如果有,则install,并重复运行安装运行安装...直到不报错为止。
或者使用tex编辑器(\TeX Studio)并使用xelatex手动编译media/Tex文件夹中的.tex文件,查看是否有包没有安装

对于没有1.2.问题却依旧报错的,可以选择重新安装新版MiK\TeX 或者安装\TeX Live-full版\\

\textbf{II.若安装的\TeX 发行版为\TeX Live}
考虑 1.\TeX Live有关路径是否添加到环境变量中 2.安装的是否为full版本\\

\textbf{III.若安装的\TeX 发行版不为以上两款}
\ 建议换成\TeX Live-full版或者MiK\TeX

\subsubsection*{Q3: 报错在文件夹内找不到svg文件}
清空media/Tex文件夹内全部内容,再次运行带文字的场景,查看Tex文件夹中的内容。

\textbf{I.若仅有tex文件和log文件}\ 按照\textbf{Q2}中方法处理。

\textbf{II.若含有xdv文件但没有svg文件}\ 考虑 1.dvisvgm是否添加到环境变量,可以使用\texttt{dvisvgm --version}观察是否
有报错来检查 2.dvisvgm版本是否过低,若\texttt{dvisvgm --version}的输出版本号为1开头,请更换新版dvisvgm(上网下载、
或者使用群文件中的版本),并注意将含dvisvgm的文件夹添加到环境变量中

\subsection{中文显示问题}
\subsubsection*{Q1: 含有中文的TextMobject编译报错Latex error converting to dvi}
将manimlib/constants.py中的\texttt{TEX\_USE\_CTEX}改成\texttt{True}再尝试

\subsubsection*{Q2: 英文可以正常显示,中文不报错,但不显示}
考虑使用的是否为\texttt{TextMobject}而不是\texttt{TexMobject}

\subsection{文字问题}
\subsubsection*{Q1: \texttt{TextMobject}和\texttt{TexMobject}有什么区别}
\texttt{TextMobject}和\texttt{TexMobject}使用的都是\LaTeX 语法。其中\texttt{TextMobject}文字模式
相当于直接在\LaTeX 环境下书写,\texttt{TexMobject}公式模式使用的是\LaTeX 的\texttt{$\backslash$begin\{align*\}}环境
或者可以看成加了\$\ \$的环境

使用\texttt{TextMobject}与\texttt{TexMobject}书写公式时
$$
\texttt{TextMobject("}\text{文字}\texttt{\$}\text{公式}\texttt{\$")} \iff \texttt{TexMobject("}\mathtt{\backslash\backslash}
\texttt{text\{}\text{文字}\texttt{\}}\text{公式}\texttt{")}
$$

\subsubsection*{Q2: \texttt{TextMobject}中怎么改字体}
\texttt{TextMobject}中只能使用\LaTeX 的内置字体族和字体形状,包括\texttt{$\backslash$textrm\{罗马字体\}}\ 
\texttt{$\backslash$textsf\{无衬线字体\}}\ \texttt{$\backslash$texttt\{打字机字体\}},
和\texttt{$\backslash$textup\{直立形状\}}\ \texttt{$\backslash$textit\{意大利体形状\}}\ 
\texttt{$\backslash$textsl\{倾斜形状\}}\ \texttt{$\backslash$textsc\{小型大写\}}

\subsubsection*{Q3: 想自定义字体怎么办}
使用新版manim特有的\texttt{Text()}类,方法如下\texttt{Text("}文字\texttt{", font="}字体\texttt{")},
其中字体要填写在计算机内存储的格式(例如\texttt{Microsoft YaHei},\texttt{Source Han Sans CN}),但是
不能使用\LaTeX 语法书写公式

\subsubsection*{Q4: 想用自定义字体写公式怎么办}
可以使用群文件里cigar666编写的\texttt{MyText()}类{\tiny Cigar牛逼}

\subsubsection*{Q5: \texttt{TexMobject}中换行是什么}
四个右划线$\backslash\backslash\backslash\backslash$,Python转义右划线,所以涉及到$\backslash$的均要写成两个
$\backslash\backslash$,而换行在\LaTeX 中是两个右杠线,所以要写成四个(或者在字符串前加r,正常书写)

\subsubsection*{Q6: 公式怎么对齐}
\textbf{方法I.}\ 直接在\texttt{TexMobject}中使用\&对齐

\textbf{方法II.}\ 两个\texttt{mobject}对齐,使用\texttt{obj2.next\_to(obj1, DOWN, aligned\_edge=LEFT)}使
\texttt{obj2}在\texttt{obj1}下方,并且左对齐

\textbf{方法III.}\ \texttt{VGroup}内对齐,使用\texttt{group.arrange(DOWN, aligned\_edge=LEFT)}使
\texttt{VGroup}中的子元素依次向下排开,并左对齐

\subsubsection*{Q6: manim写公式的示例}
\url{https://github.com/Elteoremadebeethoven/AnimationsWithManim/blob/master/English/3\_text\_like\_arrays/scenes.md}

\subsubsection*{Q7: \texttt{TexMobject}上色问题的处理办法}
\textbf{方法I.}\ 将上色的字符分开,使用\texttt{text[i].set\_color(color)}来上色

\textbf{方法II.}\ 将上色的字符分开,使用\texttt{text.set\_color\_by\_tex\_to\_color\_map(t2c)}传入
t2c字典来对相同的字符串上色

\textbf{方法III.}\ 只传入一个字符串,但同时传入\texttt{tex\_to\_color\_map=t2c}来自动拆分上色(容易出问题)

\textbf{方法IV.}\ 只传入一个字符串,使用\texttt{text[0][i]}来对细小的路径上色(一般是一个字符一个下标)

\subsubsection*{Q8: \texttt{TexMobject}的下标怎么分析}
创建函数
\begin{lstlisting}
def debugTeX(self, texm):
    for i, j in zip(range(100), texm):
        tex_id = TextMobject(str(i)).scale(0.3).set_color(PURPLE)
        tex_id.move_to(j)
        self.add(tex_id)
\end{lstlisting}
在使用时先\texttt{self.add(tex)}然后再\texttt{debugTeX(self, tex)},导出最后一帧(-s选项)。
观察每段字符上的标号,即为下标

\subsubsection*{Q9: \texttt{TexMobject}使用\texttt{$\backslash$frac}拆分时出错}
这个是Grant写\texttt{tex\_file\_writing.py}的一个bug,建议使用\texttt{\{}分子\texttt{$\backslash$over}
分母\texttt{\}}来代替\texttt{$\backslash$frac\{}分子\texttt{\}\{}分母\texttt{\}}


\subsection{素材引用问题}
\subsubsection*{Q1: 使用\texttt{SVGMobject}找不到svg文件}
\textbf{方法I.}\ 直接使用绝对路径引用svg文件

\textbf{方法II.}\ 将svg文件放到\texttt{assets/svg\_images/}文件夹中

\subsubsection*{Q2: 如何使用jpg或者png文件}
\textbf{方法I.}\ 直接使用绝对路径引用,并使用\texttt{ImageMobject}

\textbf{方法II.}\ 将jpg/png文件放到\texttt{assets/raster\_images/}文件夹中

\section{其他问题}
\subsubsection*{Q1: 没有manim源码}
最好不要使用\texttt{pip install manimlib}来装manim,请在GitHub上clone下来manim的全部内容

\subsubsection*{Q2: 群友用的manim都是什么版本}
manim不看版本,一般使用的都是最新版code。release里面带版本号的都可以看作旧版

\subsubsection*{Q3: 如何使用傅里叶级数作图}
套用Grant写好的文件
\begin{align*}
&\texttt{active\_projects/diffyq/part2/fourier\_series.py}\\
&\texttt{active\_projects/diffyq/part4/fourier\_series\_scenes.py}\\
&\texttt{active\_projects/diffyq/part4/long\_fourier\_series.py}
\end{align*}
只需要更换svg素材即可(自己制作,或者使用群里的svg素材)

\subsubsection*{Q4: svg用什么软件制作}
\texttt{Adobe Illustrator}(简称AI)或者\texttt{inkscape}(简称ink)。尽量不要使用网页版编辑器

\subsubsection*{Q5: 一些比较复杂,操纵东西比较多的动画怎么做}
使用外部剪辑软件,例如\texttt{Adobe Premiere Pro}或者达芬奇

\subsubsection*{Q5: 有什么教程}
\textbf{教程1.}\ 群主cigar666的B站专栏
\url{https://www.bilibili.com/read/readlist/rl82339}

\textbf{教程2.}\ pdcxs大大转载的manim教程
\url{https://www.bilibili.com/video/av64023740}配合源码
\url{https://github.com/Elteoremadebeethoven/AnimationsWithManim}

\textbf{教程3.}\ GitHub上cai-hust的中文教程
\url{https://github.com/cai-hust/manim-tutorial-CN}

\textbf{教程4.}\ manim源码

\section{注意}
如果有以上之外的问题,可以在群里提出,或者按照下图操作
\begin{figure}[h]
\centering
\includegraphics[width=0.35\textwidth]{Grant.png}
\end{figure}

也请注意群规第3,4条
\begin{itemize}
	\item 3. 虽为manim交流群,但不要一有问题就提出来,简单的问题能自己解决最好,不能解决时再寻求帮助
	\item 4.群主和管理员平时较忙,有时若不能及时回复敬请谅解
\end{itemize}


\end{document}