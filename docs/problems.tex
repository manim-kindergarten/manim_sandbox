%\documentclass[a4paper]{article}
\documentclass[cn,blue,14pt,normal]{elegantnote}
%\usepackage[UTF8]{ctex}
%\usepackage[colorlinks, linkcolor=blue]{hyperref}
%\usepackage[a4paper, top=2.5cm, bottom=2.5cm, left=2.5cm, right=2.5cm]{geometry}
\usepackage{tcolorbox, booktabs, fontspec, tikz, harmony}
\title{\texttt{manim}常见问题}

\author{鹤翔万里\& catfish}
\institute{\textsc{manim-kindergarten}}

\version{3.1}
\date{\zhtoday}

\lstset{
	basicstyle=\ttfamily
}

\begin{document}
	
\maketitle

\centerline{
	\includegraphics[height=3\baselineskip]{assets/Logo.png}
}

\newpage


\section*{一切之前}

\texttt{manim}是一个使用python制作视频的动画引擎。学习它你首先要会一点python,至少要学会python的基础语法、
模块的调用以及类的基础知识(如果想要阅读源码,还需要掌握更多python面向对象的知识)。没有python的知识
学\texttt{manim}是毫无意义的,会非常吃力,也会遇到非常多的问题。学会了python之后便可以少走非常多的弯路,
这也是我们的忠告。\quad 因此,加入我们
\textsc{Manim-Kindergarten}\footnote{QQ群:862671480}首先要会使用python,我们会在入群问
题\footnote{\url{https://b23.tv/KmAvsG}}中进行一个最基础的检测。

\textbf{\underline{最后,本常见问题文档基于\texttt{manim}的\texttt{cairo-backend}分支版本\footnote{即旧版\texttt{manim},原\texttt{master}分支版本},}}\\
\textbf{\underline{不包含新版\texttt{manim}中出现的新问题。}}

\subsection*{如何提问:}

\begin{enumerate}[I.]
	\item 在mk群里提问,首先需要阅读完本常见问题文档。
	
	如果群友对你说“常见问题”或“RTFM”,那么就说明你问的问题在本文档里已经有明确的解答。
	
	\item 确保你问的问题不是由于python基础语法问题而造成的错误。
	
	如,不要问\texttt{IndentationError},尽量不要问由于拼写导致的错误等。当群友劝你学习python
	或者说“STFW”时,那么就说明你的问题属于没有掌握python基础语法而导致的问题,而且你可以在网络上
	轻松搜索到解决方案。(但这时更建议你认真打牢python基础)
	
	\item 如果在\texttt{manim}使用过程中出现了报错:
	
	\begin{enumerate}[1.]
		\item 确保问题不在本文档中
		\item 将你的代码和\textbf{完整的}报错信息全部发送到群中,这样会方便群友为你解答
	\end{enumerate}

	\item 如果你想要实现某个效果,但不知道如何操作。请将你想要达到的效果详细易懂地描述出来。
	\item 语气友善,态度谦虚,避免造成不必要的纠纷。
\end{enumerate}

\subsection*{教程推荐:}

\begin{enumerate}[I.]
	\item Python教程:
	
	\begin{enumerate}[1.]
		\item \href{https://www.ituring.com.cn/book/1861}{《Python编程:从入门到实践》},ISBN 978-7-115-42802-8
		
		\item \href{https://www.ituring.com.cn/book/1564}{《流畅的Python》},ISBN 978-7-115-45415-7(适合掌握Python基础知识后进阶)
		
		\item 菜鸟教程-Python3教程 \url{https://www.runoob.com/python3/python3-tutorial.html}
	\end{enumerate}

	\item \texttt{manim}教程:
	
	\begin{enumerate}[1.]
		\item \texttt{manim}教程文档(制作中):\url{https://manim.ml/}
		
		\item \texttt{MK}制作的系列视频教程(制作中)
		\begin{itemize}
			\item \url{https://space.bilibili.com/171431343/favlist?fid=947158443}
		\end{itemize}
	
		\item \texttt{MK}制作的视频源码(videos/)和常用自定义类(utils/)
	
		\item 群主\texttt{cigar666}的B站专栏
		\begin{itemize}
			\item \url{https://www.bilibili.com/read/readlist/rl82339}
		\end{itemize}
	
		\item \texttt{pdcxs}大大转载的\texttt{manim}教程
		\begin{itemize}
			\item \url{https://www.bilibili.com/video/av64023740}
			\item 源码 \url{https://github.com/Elteoremadebeethoven/AnimationsWithManim}
		\end{itemize}
	
		\item \texttt{GitHub}上\texttt{cai-hust}的中文教程
		\begin{itemize}
			\item \url{https://github.com/cai-hust/manim-tutorial-CN}
		\end{itemize}
	
		\item 看\texttt{manim}源码
	\end{enumerate}

	\item 新版\texttt{manim}教程:
	
	\begin{enumerate}[1.]
		\item 新版\texttt{manim}官方文档:\url{https://3b1b.github.io/manim/}
		
		\item 新版\texttt{manim}中文文档:\url{https://manim.ml/shaders/}
	\end{enumerate}

	\item OpenGL及shaders教程(Grant亲自推荐)
	\begin{enumerate}[1.]
		\item The Book of Shaders: \url{https://thebookofshaders.com/}
		
		\item Python \& OpenGL for Scientific Visualization: \url{https://www.labri.fr/perso/nrougier/python-opengl/}
		
		\item Geometry Shader: \url{https://learnopengl.com/Advanced-OpenGL/Geometry-Shader}
	\end{enumerate}
\end{enumerate}

\newpage

\tableofcontents

\newpage

\section{安装问题}

安装时最好不要看\texttt{README.md}自己研究,
推荐一视数学卷毛杨的两个教程,和教程文档中的安装指南\url{https://manim.ml/installation}:
\begin{itemize}
    \item \url{https://www.bilibili.com/video/av38126904}
    \item \url{https://www.bilibili.com/read/cv4139851}
\end{itemize}

\subsection{\texttt{Python}问题}
\subsubsection*{Q1: 使用\texttt{anaconda},命令行输入\texttt{python}无反应或报错}

考虑\texttt{path}环境变量是否填全\footnote{安装\texttt{anaconda}时是否勾选添加到\texttt{path}变量},\texttt{path}变量里应该有:
\begin{lstlisting}[frame=none, columns=flexible]
    <your_path>\Anaconda3;
    <your_path>\Anaconda3\Scripts;
    <your_path>\Anaconda3\Library\bin;
\end{lstlisting}

\subsubsection*{Q2: \texttt{pip install ...}时满屏红字报错,或者安装过慢}

更换国内镜像源,使用 
\begin{lstlisting}[frame=none, columns=flexible]
    pip install -r requirements.txt -i https://pypi.tuna.tsinghua.edu.cn/simple
\end{lstlisting}

代替\footnote{临时换源}
\begin{lstlisting}[frame=none, columns=flexible]
    pip install -r requirements.txt
\end{lstlisting}

\subsubsection*{Q3: \texttt{pip}安装\texttt{pycairo}总是失败}

下载\texttt{pycairo}对应版本的\texttt{whl}包
\footnote{可在\url{https://www.lfd.uci.edu/~gohlke/pythonlibs/\#pycairo}中下载,注意\texttt{Python}版本和系统版本是否均合适}
并手动安装
\begin{lstlisting}[frame=none, columns=flexible]
    pip install pycairo......whl
\end{lstlisting}

\subsubsection*{Q4: \texttt{pip}安装过包,但运行时提示没有模块}
考虑电脑上是否有多个\texttt{Python},确定\texttt{pip}把包装到了需要使用的\texttt{Python}上面。

\subsubsection*{Q5: 关于\texttt{scipy}有报错}
可能是版本不对,使用\texttt{pip uninstall scipy}后重新\texttt{pip install scipy}

\newpage

\section{运行时问题}

\begin{note}
	在出现以下问题时,请确保你正在使用最新版\texttt{cairo-backend}分支的\texttt{manim}
\end{note}

\subsection{\texttt{import}问题}
\subsubsection*{Q1: 没有模块\texttt{big\_ol\_pile\_of\_manim\_imports}}

将文件中的
\begin{lstlisting}[frame=none, columns=flexible]
    from big_ol_pile_of_manim_imports import *
\end{lstlisting}

改成
\begin{lstlisting}[frame=none, columns=flexible]
    from manimlib.imports import *
\end{lstlisting}

\subsubsection*{Q2: 缺少模块\texttt{pygments}\footnote{已在\href{https://github.com/3b1b/manim/pull/1147}{\#1147}中修复}}

手动安装 \texttt{pip install pygments}

\subsection{\LaTeX 问题}
\subsubsection*{Q1: 报错\texttt{Latex error converting to dvi}}
先不要管错误在哪,先把\texttt{manimlib/constants.py}中的\texttt{TEX\_USE\_CTEX}改成\texttt{True}再运行

\subsubsection*{Q2: 报错 \texttt{xelatex error converting to xdv}}\label{sub:Q2}
若为\texttt{Windows}系统,先把\texttt{manimlib/constants.py}的第29行:
\begin{lstlisting}[frame=none, columns=flexible]
    MEDIA_DIR = "./media"
\end{lstlisting}

改成\footnote{已在\href{https://github.com/3b1b/manim/pull/689}{\#689}中修复}
\begin{lstlisting}[frame=none, columns=flexible]
    MEDIA_DIR = os.path.join(os.getcwd(), "media")
\end{lstlisting}

再进行尝试。如果仍然出错,尝试将\texttt{ctex\_template.tex}中的\texttt{\textbackslash usepa}\\
\texttt{ckage\{ctex\}}提到该文件的第二行再进行尝试\footnote{已在\href{https://github.com/3b1b/manim/pull/1187}{\#1187}中修复}。
还出错误的话,向下继续按步骤进行:

\begin{enumerate}[I.]
    \item \textbf{若安装的\TeX 发行版为MiK\TeX}
    \begin{enumerate}[1.]
        \item MiK\TeX 的有关路径是否添加到环境变量中
        \item 是否有包没有装全
    \end{enumerate}

    \begin{tcolorbox}
        对于\texttt{2.},可以正常运行一遍\texttt{WriteStuff}场景,看是否有框弹出提示\texttt{install}什么东西,
        如果有,则\texttt{install},并重复运行安装运行安装...直到不报错为止。 \\
        或者使用\TeX 编辑器\TeX Studio等
        并使用\texttt{xelatex}手动编译\texttt{media/Tex}文件夹中的\texttt{.tex}文件,查看是否有包没有安装。
    \end{tcolorbox}
        
    \begin{tcolorbox}
        对于没有\texttt{1.}和\texttt{2.}问题却依旧报错的,可以选择重新安装新版MiK\TeX 或者安装\TeX Live-full版(推荐)。
    \end{tcolorbox}

    \item \textbf{若安装的\TeX 发行版为\TeX Live}
    \begin{enumerate}[1.]
        \item \TeX Live有关路径是否添加到环境变量中
        \item 安装的是否为\texttt{full}版本
    \end{enumerate}

    \item \textbf{若安装的\TeX 发行版不为以上两款}
    
    建议换成\TeX Live-full版或者MiK\TeX,并且注意在重新安装前删除旧版
\end{enumerate}

\subsection{dvisvgm问题}

\subsubsection*{Q1: 报错\texttt{OSError: No file matching .svg in image directory}}

清空\texttt{media/Tex}文件夹内全部内容,再次运行带文字的场景,查看\texttt{Tex}文件夹中的内容:

\begin{enumerate}[I.]
    \item 若含有\texttt{.tex}文件,但没有\texttt{.xdv}文件,按照\texttt{\ref{sub:Q2}}中方法处理
    \item 若含有\texttt{.xdv}文件但没有\texttt{.svg}文件
    \begin{enumerate}[1.]
        \item 检查\texttt{divsvgm}是否添加到环境变量,可以使用\texttt{dvisvgm -\!-versi}
        
        \texttt{on}观察是否由报错来检查
        \item \texttt{dvisvgm}版本是否过低,若\texttt{dvisvgm -\!-verison}的输出版本号小于2.4,请更换新版\texttt{dvisvgm}\footnote{上网下载、或者使用群文件中的版本},并注意将含有\texttt{dvisvgm}的文件夹添加到环境变量中
        \item 若\texttt{dvisvgm}的版本高于2.4,可能是你的\texttt{dvisvgm}暂不支持\texttt{PostS}
        
        \texttt{cript}请按照Q2中指导操作
    \end{enumerate}
\end{enumerate}

\subsubsection*{Q2: 如何让\texttt{dvisvgm}支持\texttt{PostScript}\footnote{这部分解决方案来自\texttt{dvisvgm}的FAQ:\url{https://dvisvgm.de/FAQ/}}}

打开终端,输入\texttt{dvisvgm -l}检查有没有\texttt{ps         dvips PostScript spe}\\
\texttt{cials}(如果有,则已经支持了\texttt{PostScript});输入\texttt{dvisvgm -h}
检查有没有\texttt{-\!-libgs=filename}。接下来按照以下处理

\begin{enumerate}[1.]
	\item 若\texttt{dvisvgm -h}输出中没有\texttt{-\!-libgs},\texttt{dvisvgm -l}中没有\texttt{ps}
	\begin{itemize}
		\item 你安装的\texttt{dvisvgm}无法支持\texttt{PostScript},请换一个安装再试
	\end{itemize}
	\item 若\texttt{dvisvgm -h}输出中含有\texttt{-\!-libgs}
	
	这说明你的\texttt{dvisvgm}需要\texttt{Ghostscript}才能支持\texttt{PostScript},按下述操作:
	\begin{enumerate}[a.]
		\item 查找\texttt{Ghostscript}库
		\begin{itemize}
			\item 如果是Windows32位系统,则需要\texttt{gsdll32.dll}(可能位于\texttt{C:\textbackslash Windows\textbackslash System32\textbackslash}文件夹中)
			
			\item 如果是Windows64位系统,则需要\texttt{gsdll64.dll}(可能位于\texttt{C:\textbackslash Windows\textbackslash System32\textbackslash}文件夹中)
			
			\item 如果是Linux系统,则需要\texttt{libgs.so}(可能位于\texttt{/usr/local/li}
			\texttt{b/}或\texttt{/usr/lib/}文件夹中)
			
			\item 如果是MacOS系统,则需要\texttt{libgsl.dylib}(可能位于\texttt{/usr/lo}
			\texttt{cal/lib/}或\texttt{/opt/local/lib/}文件夹中)
		\end{itemize}
		\item 添加\texttt{Ghostscript}库,可以通过以下三种方法:
		\begin{itemize}
			\item 把上述文件位于的文件夹添加到\texttt{PATH}环境变量中
			\item 把上述文件的完整路径(包括目录和文件名)设置为\texttt{LIBGS}环境变量的值
			\item 终端输入\texttt{dvisvgm -\!-libgs="文件完整位置(包括文件名)"}
		\end{itemize}
	\end{enumerate}
\end{enumerate}

如上操作后,再输入\texttt{dvisvgm -l},如果含有\texttt{ps}则成功支持了\texttt{PostSc}\\
\texttt{ript}。

\subsection{中文显示问题}
\subsubsection*{Q1: 含有中文的\texttt{TextMobject}编译报错,\texttt{Latex error converting to dvi}}

将\texttt{manimlib/constants.py}中的\texttt{TEX\_USE\_CTEX}改成\texttt{True}再尝试

\subsubsection*{Q2: 英文可以正常显示,中文不报错,但不显示}

考虑使用的是否为\texttt{TextMobject}而不是\texttt{TexMobject}

\subsection{文字问题}
\subsubsection*{Q1: \texttt{TextMobject}和\texttt{TexMobject}有什么区别}

\texttt{TextMobject}和\texttt{TexMobject}使用的都是\LaTeX 语法

其中\texttt{TextMobject}文字模式相当于直接在\LaTeX 环境下书写

\texttt{TexMobject}公式模式使用的是\LaTeX 的 \texttt{\textbackslash begin\{align*\}}
环境或者可以看成加了$\texttt{\$}\texttt{\$}$的环境

使用\texttt{TextMobject}与\texttt{TexMobject}书写公式时:

\noindent \fbox{$\texttt{TextMobject("} \text{文字} \texttt{\$} \text{公式} \texttt{\$")} \Longleftrightarrow \texttt{TexMobject("\textbackslash \textbackslash text\{} \text{文字} \texttt{\}} \text{公式} \texttt{")}$}

\subsubsection*{Q2: \texttt{TextMobject}中怎么改字体样式}

\texttt{TextMobject}中只能使用\LaTeX 的字体样式

字体常用样式命令见表:
\begin{table}[htbp]
    \centering
    \begin{tabular}{llll}
        \toprule
        字体样式 & \LaTeX 代码  & 字体样式 & \LaTeX 代码 \\
        \midrule
        \textrm{roman}  & \texttt{\textbackslash textrm\{\dots\}} & \textbf{bold face} & \texttt{\textbackslash textbf\{\dots\}} \\
        \textsf{sans serif} & \texttt{\textbackslash textsf\{\dots\}} & \textmd{medium weight} & \texttt{\textbackslash textmd\{\dots\}} \\
        \texttt{typewriter} & \texttt{\textbackslash texttt\{\dots\}} & \textit{italic} & \texttt{\textbackslash textit\{\dots\}} \\
        \textsc{Small Caps} & \texttt{\textbackslash textsc\{\dots\}} & \textsl{slanted} & \texttt{\textbackslash textsl\{\dots\}} \\
        \textup{upright} & \texttt{\textbackslash textup\{\dots\}} \\
        \bottomrule
    \end{tabular}
\end{table}

严格地讲中文字体并没有衬线、无衬线、等宽、斜体等概念

\subsubsection*{Q3: 想自定义字体怎么办}

使用新版\texttt{manim}特有的\texttt{Text()}类,
方法如下$\texttt{Text("}\text{文字}\texttt{", font="}$
$\text{字体}\texttt{")}$,
其中字体要填写在计算机内存储的格式\footnote{例如:Microsoft YaHei,Source Han Sans CN(Windows可以打开C:/Windows/Fonts中的字体文件查看名称)},但是不能使用\LaTeX 语法书写公式

\subsubsection*{Q4: 想用自定义字体写公式怎么办}

可以使用\texttt{cigar666}编写的\texttt{MyText()}类,源码地址:\url{https://github.com/manim-kindergarten/manim_sandbox/blob/master/utils/mobjects/MyText.py}

\subsubsection*{Q5: \texttt{TexMobject}中换行是什么}
四个右划线\texttt{\textbackslash \textbackslash \textbackslash \textbackslash},
\texttt{Python}转义右划线,所以涉及到\texttt{\textbackslash}的均要写成两个\texttt{\textbackslash \textbackslash},
而换行在\LaTeX 中是两个右划线,所以要写成四个\footnote{或者在字符串前加r,正常书写}

\subsubsection*{Q6: 公式怎么对齐}
\begin{enumerate}[I.]
    \item 直接在\texttt{TexMobject}中使用\texttt{\&}对齐
    \item 两个\texttt{mobject}对齐,使用\texttt{obj2.next\_to(obj1, DOWN, aligned\_edge=LEFT)}使\texttt{obj2}在\texttt{obj1}下方,并左对齐
    \item \texttt{VGroup}内对齐,使用\texttt{group.arrange(DOWN, aligned\_edge=LEFT)}使\texttt{VGroup}中的子元素依次向下排开,并左对齐
\end{enumerate}

写公式的示例:

\url{https://github.com/Elteoremadebeethoven/AnimationsWithManim/blob/master/English/3_text_like_arrays/scenes.md}

\subsubsection*{Q7: \texttt{TexMobject}上色问题的处理办法}
\begin{enumerate}[I.]
    \item 将上色的字符分开,使用\texttt{text[i].set\_color(color)} 来上色
    \item 将上色的字符分开,使用\texttt{text.set\_color\_by\_tex\_to\_color\_map(t2c)}传入\texttt{t2c}字典来对相同的字符串上色
    \item 只传入一个字符串,但同时传入\texttt{tex\_to\_color\_map=t2c}来自动拆分上色(容易出问题)
    \item 只传入一个字符串,使用\texttt{text[0][i]}来对细小的路径上色(一般是一个字符一个下标)
\end{enumerate}

\subsubsection*{Q8: \texttt{TexMobject}的下标怎么分析}

\begin{enumerate}[I.]
	\item 使用\texttt{debugTeX}\footnote{\url{https://github.com/manim-kindergarten/manim\_sandbox/blob/master/utils/functions/debugTeX.py}},先\texttt{self.add(tex)}然后再\texttt{debugTeX(self, tex)},
	导出最后一帧\footnote{-s 选项},观察每段字符上的标号,即为下标
	\item 使用自带的函数\texttt{get\_submobject\_index\_labels}获取下标的\texttt{VGroup},然后添加
\end{enumerate}

关于\texttt{Tex(t)Mobject}的结构,详细可以看视频\url{https://www.bilibili.com/video/BV1CC4y1H7kp}

\subsubsection*{Q9: \texttt{TexMobject}使用 \texttt{\textbackslash frac} 拆分时出错}
这个是\texttt{Grant}写\texttt{tex\_file\_writing.py} 的一个\texttt{bug},
建议使用\texttt{\{}分子 \texttt{\textbackslash over}分母\texttt{\}}
来代替 \texttt{\textbackslash frac\{}分子\texttt{\}\{}分母\texttt{\}}

\subsubsection*{Q10: 使用\texttt{\textbackslash left\textbackslash\{ ...\textbackslash right.}报错}
\begin{lstlisting}[frame=none, columns=flexible]
    TexMobject(r"\left\{\begin{matrix} a+b \\ b+a \\ \end{matrix}\right.")
\end{lstlisting}

\texttt{matrix}这样的写法在\texttt{manim}中会报错,无法生成\texttt{dvi},
原因是\texttt{manim}会自动寻找相对应的括号来匹配,但这里并没有右大括号,而是\texttt{.}

所以推荐使用\texttt{cases}环境,效果是一样的:$\begin{cases}
    a+b \\
    b+a \\
\end{cases}$
\begin{lstlisting}[frame=none, columns=flexible]
    TexMobject(r"\begin{cases} a+b \\ b+a \\ \end{cases}")
\end{lstlisting}

\subsection[素材引用问题]{素材引用问题\footnote{关于插入素材(图片),详细可以看视频\url{https://www.bilibili.com/video/BV1CC4y1H7kp}}}
\subsubsection*{Q1: 使用\texttt{SVGMobject}找不到\texttt{svg}文件}
\begin{enumerate}[I.]
    \item 直接使用绝对路径引用\texttt{svg}文件
    \item 将\texttt{svg}文件放到\texttt{assets/svg\_images/}文件夹中
\end{enumerate}

\subsubsection*{Q2: 如何使用\texttt{jpg}或者\texttt{png}文件}
\begin{enumerate}[I.]
    \item 直接使用绝对路径引用,并使用\texttt{ImageMobject}
    \item 将\texttt{jpg/png}文件放到\texttt{assets/raster\_images/}文件夹中
\end{enumerate}

\subsubsection*{Q3: 能否导入\texttt{gif}文件}
可以使用\texttt{ImageMobject}导入,但是只保留第一帧,不会显示动图



\newpage

\section{其它问题}

\subsubsection*{Q1: 没有\texttt{manim}源码}
\addcontentsline{toc}{subsection}{Q1: 没有\texttt{manim}源码}
最好不要使用\texttt{pip install manimlib}来装\texttt{manim},请在\texttt{GitHub}上\texttt{clone}下来\texttt{manim}的全部内容

\subsubsection*{Q2: 群友用的\texttt{manim}都是什么版本}
\addcontentsline{toc}{subsection}{Q2: 群友用的\texttt{manim}都是什么版本}
一般使用的都是\texttt{GitHub}上的最后一版\texttt{cairo-backend}分支上的源码,少部分使用的是\texttt{master}
分支上的新版本源码(不在本文档讨论范围内)

\subsubsection*{Q3: 如何使用傅里叶级数作图}
\addcontentsline{toc}{subsection}{Q3: 如何使用傅里叶级数作图}
套用 Grant 写好的文件 (有部分代码\texttt{import}部分路径不对,请自行调整)
\begin{lstlisting}[frame=none, columns=flexible]
    from_3b1b/active/diffyq/part2/fourier_series.py
    from_3b1b/active/diffyq/part4/fourier_series_scenes.py
    from_3b1b/active/diffyq/part4/long_fourier_series.py
\end{lstlisting}

\subsubsection*{Q4: 傅里叶级数作图如何调整时长}
\addcontentsline{toc}{subsection}{Q4: 傅里叶级数作图如何调整时长}
\texttt{CONFIG}中\texttt{run\_time}无法控制,使用\texttt{slow\_factor}和\texttt{n\_cycles}来控制

$\mathtt{\dfrac{1}{slow\_factor}}$为一个循环的时间,\texttt{n\_cycles}为循环的个数


只需要更换\texttt{svg}素材即可\footnote{自己制作,或者使用这里的\texttt{svg}素材:\url{https://github.com/manim-kindergarten/manim_sandbox/tree/master/assets/svg_images}}

\subsubsection*{Q5: \texttt{svg}用什么软件制作}
\addcontentsline{toc}{subsection}{Q5: \texttt{svg}用什么软件制作}
\texttt{Adobe Illustrator}(简称 AI,推荐)或者\texttt{inkscape}(简称 ink,不推荐)。而且不要使用网页版编辑器

目前\texttt{manim}对\texttt{SVG}的解析很局限,推荐使用\texttt{AI}\footnote{并且使用“另存为$\to$SVG”的方式,不要使用导出}

\subsubsection*{Q6: 动画怎么显示旋转一个物体}
\addcontentsline{toc}{subsection}{Q6: 动画怎么显示旋转一个物体}
使用\texttt{Ratate}和\texttt{Rotating},区别在群文件中有视频

\subsubsection*{Q7: \texttt{Transform}和\texttt{ReplacementTransform}有什么区别}
\addcontentsline{toc}{subsection}{Q7: \texttt{Transform}和\texttt{ReplacementTransform}有什么区别}
\begin{enumerate}[1.]
	\item \texttt{Transform(A, B)}在画面上\texttt{A}变成了\texttt{B}的样子,但是画面上的物体名字还叫\texttt{A}
	\item \texttt{ReplacementTransform(A, B)}在画面上\texttt{A}变成了\texttt{B}的样子,并且画面上的物体名字叫\texttt{B}
\end{enumerate}

所以以下两个效果相同
\begin{lstlisting}[frame=none, columns=flexible]
self.play(Transform(A, B))
self.play(Transform(A, C))
\end{lstlisting}

\begin{lstlisting}[frame=none, columns=flexible]
self.play(ReplacementTransform(A, B))
self.play(ReplacementTransform(B, C))
\end{lstlisting}

\subsubsection*{Q8: 怎么控制物体移动或者\texttt{Transform}的速率}
\addcontentsline{toc}{subsection}{Q8: 怎么控制物体移动或者\texttt{Transform}的速率}
使用\texttt{rate\_func},一些\texttt{manim}中已经定义的在群文件中有视频

\begin{figure}[htbp]
    \begin{minipage}{0.18\linewidth}
        \centerline{\includegraphics[width=1in]{assets/linear.png}}    
    \end{minipage}
    \begin{minipage}{0.18\linewidth}
        \centerline{\includegraphics[width=1in]{assets/lingering.png}}    
    \end{minipage}
    \begin{minipage}{0.18\linewidth}
        \centerline{\includegraphics[width=1in]{assets/result.png}}    
    \end{minipage}
    \begin{minipage}{0.18\linewidth}
        \centerline{\includegraphics[width=1in]{assets/running_start.png}}    
    \end{minipage}
    \begin{minipage}{0.18\linewidth}
        \centerline{\includegraphics[width=1in]{assets/rush_from.png}}    
    \end{minipage}
\end{figure}

\begin{figure}[htbp]
    \begin{minipage}{0.18\linewidth}
        \centerline{\includegraphics[width=1in]{assets/rush_into.png}}    
    \end{minipage}
    \begin{minipage}{0.18\linewidth}
        \centerline{\includegraphics[width=1in]{assets/slow_into.png}}    
    \end{minipage}
    \begin{minipage}{0.18\linewidth}
        \centerline{\includegraphics[width=1in]{assets/smooth.png}}    
    \end{minipage}
    \begin{minipage}{0.18\linewidth}
        \centerline{\includegraphics[width=1in]{assets/wiggle.png}}    
    \end{minipage}
    \begin{minipage}{0.18\linewidth}
        \centerline{\includegraphics[width=1in]{assets/there_and_back.png}}    
    \end{minipage}
\end{figure}

\begin{figure}[htbp]
    \begin{minipage}{0.18\linewidth}
        \centerline{\includegraphics[width=1in]{assets/double_smooth.png}}    
    \end{minipage}
    \begin{minipage}{0.2\linewidth}
        \centerline{\includegraphics[width=1.2in]{assets/exponential_decay.png}}    
    \end{minipage}
    \begin{minipage}{0.3\linewidth}
        \centerline{\includegraphics[width=1.8in]{assets/there_and_back_with_pause.png}}    
    \end{minipage}
\end{figure}

\subsubsection*{Q9: 数学符号/公式 用\LaTeX 怎么打}
\addcontentsline{toc}{subsection}{Q9: 数学符号\texttt{/}公式 用\LaTeX 怎么打}
请见 \url{https://www.luogu.com.cn/blog/IowaBattleship/latex-gong-shi-tai-quan}

推荐妈咪叔维护的\url{https://www.latexlive.com/}

\subsubsection*{Q10: 一些特殊\LaTeX 的外部包}
\addcontentsline{toc}{subsection}{Q10: 一些特殊\LaTeX 的外部包}

\Ganz \Halb \Vier \Acht \Sech \Zwdr

\textbf{如何使用\texttt{manim}画出上面的音符,或怎么使用这些包?}

在\texttt{manimlib}目录下的\texttt{ctex\_template.tex}或者\texttt{tex\_template.tex}文件中
添加外部包的名称\footnote{修改\texttt{TEX\_USE\_CTEX}为\texttt{True}的,可以只在\texttt{ctex\_template.tex}中添加}

就拿上面的音符为例,因为是在\texttt{harmony}包中的,所以在\texttt{tex}文件中添加\texttt{\textbackslash usepackage\{harmony\}}\footnote{不需要使用的时候记得改回来哦\label{change}}

然后新建一个\texttt{py}文件,写入代码
\begin{lstlisting}[frame=none, columns=flexible]
    from manimlib.imports import *
    class TestHarmony(Scene):
        def construct(self):
            # harmony具体用法请百度
            harmony = TextMobject(r"\Ganz \Halb \Vier \Acht \Sech \Zwdr")
            self.play(ShowCreation(harmony))
            self.wait()
\end{lstlisting}

运行py文件即可

\subsubsection*{Q11: 使用\LaTeX 外部包,编译错误或者无显示}
\addcontentsline{toc}{subsection}{Q11: 使用\LaTeX 外部包,编译错误或者无显示}
首先,并不是所有外部包都能在\texttt{manim}中顺利使用,大多都不支持\texttt{xelatex}编译,
所以建议需要使用外部包时只用\texttt{latex}编译\footnote{即把\texttt{TEX\_USE\_CTEX}改为\texttt{False}}

至于有些群友常用\texttt{TiKZ}这个外部包,也是使用\texttt{latex}才能顺利运行,
在\texttt{xelatex}用{} \texttt{\textbackslash draw}会无法显示,
需要修改\texttt{tex\_template.tex}文件\textsuperscript{\ref{change}},修改成如下:

\begin{lstlisting}[frame=none, columns=flexible]
    \documentclass[preview, dvisvgm]{standalone}
    \usepackage{tikz}
\end{lstlisting}

新建\texttt{py}文件,写入代码来画一条线:\begin{tikzpicture}
    \draw (-1, 0) -- (1, 0);
\end{tikzpicture}

\begin{lstlisting}[frame=none, columns=flexible]
    class TestTikz(Scene):
        def construct(self):
            tikz = TextMobject(
                # tikz具体用法请百度
                r"\tikz{\draw (-1, 0) -- (1, 0);}",
                color=WHITE,
                stroke_width=1,
                stroke_opacity=1,
            )
            self.play(ShowCreation(tikz))
            self.wait()
\end{lstlisting}

运行py文件即可

\subsubsection*{Q12: 一些比较复杂,操纵东西比较多的动画怎么做}
\addcontentsline{toc}{subsection}{Q12: 一些比较复杂,操纵东西比较多的动画怎么做}
使用外部剪辑软件,例如\texttt{Adobe Premiere Pro}或者达芬奇


\subsubsection*{Q13: 一个\texttt{self.play}里写两个\texttt{ApplyMethod}只对一个起作用怎么办}
\addcontentsline{toc}{subsection}{Q13: 一个\texttt{self.play}里写两个\texttt{ApplyMethod}只对一个起作用怎么办}
去掉\texttt{ApplyMethod},例如:
\begin{lstlisting}[frame=none, columns=flexible]
    self.play(ApplyMethod(mob.scale, 2), ApplyMethod(mob.shift, DOWN))
\end{lstlisting}

改成
\begin{lstlisting}[frame=none, columns=flexible]
    self.play(mob.scale, 2, mob.shift, DOWN)
\end{lstlisting}

\newpage

\subsubsection*{Q14: 如何解决二维画面中的图层问题}
\addcontentsline{toc}{subsection}{Q14: 如何解决二维画面中的图层问题}
可以使用\texttt{pdcxs}添加的\texttt{plot\_depth},具体更改见下图\footnote{\texttt{plot\_depth}的值越大,运行出来的物体就越在上面}

\texttt{MK fork}的版本已经做了修改:\url{https://github.com/manim-kindergarten/manim}
\begin{figure}[h]
	\begin{center}
		\includegraphics[width=\linewidth]{assets/pd1.png}
	\end{center}
\begin{center}
\includegraphics[width=\linewidth]{assets/pd2.png}
\end{center}
\end{figure}



\newpage

\subsubsection*{Q15: 如何导出\texttt{gif}文件}
\addcontentsline{toc}{subsection}{Q15: 如何导出\texttt{gif}文件}
在最后一个\texttt{cairo-backend}版本中,\texttt{manim}导出\texttt{gif}已经失效,可以导出\texttt{mp4},后用\texttt{ffmpeg}转换。也可以按照下图修改源码

\texttt{MK fork}的版本已经做了修改:\url{https://github.com/manim-kindergarten/manim}
\begin{figure}[h]
	\begin{center}
		\includegraphics[width=\linewidth]{assets/gif.png}
	\end{center}
\end{figure}

改过后,在输入命令时加上\texttt{-i}选项,就能导出\texttt{gif}了

\subsubsection*{Q16: 如何导出透明的图片或者视频}
\addcontentsline{toc}{subsection}{Q16: 如何导出透明的图片或者视频}
在运行命令的时候加上 \texttt{-t}选项
\begin{itemize}
	\item 如果是 \texttt{-s}保存图片,则会存储为背景透明的\texttt{png}图片
	\item 如果是 \texttt{-l/-m/-w}保存视频,则会存储为背景透明的\texttt{mov}视频文件,方便\texttt{pr}中的剪辑
\end{itemize}

\subsubsection*{Q17: 渲染视频的画质和帧率怎么调整}
\addcontentsline{toc}{subsection}{Q17: 渲染视频的画质和帧率怎么调整}
\texttt{manim}的默认画质有四种
\begin{itemize}
	\item \texttt{-l} 最低画质 \texttt{480P15}
	\item \texttt{-m} 中等画质 \texttt{720P30}
	\item \texttt{-\!-high\_quality}\footnote{没有缩写} 高画质 \texttt{1080P60}
	\item \texttt{-w} 导出(最高)画质 \texttt{1440P60(2K)}
	\item \texttt{-uhd} 超高清 \texttt{4K120fps}(B站最高)\footnote{仅限\texttt{MK}版本\texttt{manim}}
\end{itemize}

不加画质选项,默认使用 \texttt{-w}最高画质\footnote{比如 \texttt{-p}(虽然很多人把 \texttt{-p}当成了 \texttt{-w}...)}。
可以通过修改\texttt{constants.py}中对应的画面长宽和帧率来修改\footnote{\texttt{manimlib/constants.py}的\texttt{118}行开始}

一般把 \texttt{-w}最高画质修改成\texttt{1080P60}


\subsubsection*{Q18: 有没有什么好的场景例子供学习}
\addcontentsline{toc}{subsection}{Q18: 有没有什么好的场景例子供学习}

\begin{enumerate}[1.]
	\item \texttt{GitHub}上\texttt{manim-kindergarten/manim\_sandbox}中的\texttt{demo}和\texttt{videos}文件夹中的代码
	\item \texttt{Grant}的代码\footnote{\texttt{from\_3b1b}文件夹中}对应\texttt{3B1B}的视频,可能会有报错,需要魔改
	\item 群文件里“\texttt{manim}相关的\texttt{python}代码及视频结果”
	\item 群里几个B站\texttt{up}主的\texttt{GitHub}库对应他们的代码
	\begin{itemize}
		\item \texttt{cigar666} \url{https://github.com/cigar666/my_manim_projects}
		\item 鹤翔万里 \url{https://github.com/TonyCrane/manim_projects}
		\item \texttt{pdcxs} \url{https://github.com/pdcxs/ManimProjects}
		\item 有一种悲伤叫颓废 \url{https://github.com/136108Haumea/my-manim}
	\end{itemize}
\end{enumerate}

\subsubsection*{Q19: 新版本\texttt{manim}是什么,和旧版有什么区别}
\addcontentsline{toc}{subsection}{Q19: 新版本\texttt{manim}是什么,和旧版有什么区别}

新版\texttt{manim}使用\texttt{OpenGL}和\texttt{moderngl}来进行GPU渲染,会有更快的速度,
也支持实时渲染和交互操作。更改了很多底层逻辑和结构,也调整了一些小的用法。
详细在针对新版的常见问题中叙述。

关于三个版本的\texttt{manim}的简要说明在\href{https://github.com/3b1b/manim/issues/1243}{\#1243}

\newpage

\section{注意}

如果有以上之外的问题,可以在群里提出,也可以在GitHub上提出issue,或者按照下图操作

    \begin{figure}[h]
        \begin{center}
            \includegraphics[width=6cm]{assets/grant.png}
        \end{center}
    \end{figure}

也请注意群规第 3,4 条
\begin{itemize}
    \item 3.虽为 manim 交流群,但不要一有问题就提出来,简单的问题能自己解决最好,不能解决时再寻求帮助
    \item 4.群主和管理员平时较忙,有时若不能及时回复敬请谅解
\end{itemize}

\begin{center}
\textbf{最后,祝大家好运(*^-^*)}
\end{center}

\newpage

\section{附:ChangeLog}

\subsection*{v3.0}
\begin{itemize}
	\item 使用了Elegant\LaTeX Note主题
	\item 增加了\texttt{master}分支和\texttt{shaders}分支的区别
\end{itemize}

\subsection*{v3.1}
\begin{itemize}
	\item 修复了代码段字符间距大且包含空格的bug
	\item 增加了文档首页“一切之前”部分
	\item 增加了针对新旧版\texttt{manim}的区别和新版教程链接
	\item 增加了由于\texttt{dvisvgm}问题导致\texttt{OSError}的解决方案
	\item 增加了ChangeLog部分
\end{itemize}

\end{document}