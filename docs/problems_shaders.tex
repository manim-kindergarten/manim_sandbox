%\documentclass[a4paper]{article}
\documentclass[cn,blue,14pt,normal]{elegantnote}
%\usepackage[UTF8]{ctex}
%\usepackage[colorlinks, linkcolor=blue]{hyperref}
%\usepackage[a4paper, top=2.5cm, bottom=2.5cm, left=2.5cm, right=2.5cm]{geometry}
\usepackage{tcolorbox, booktabs, fontspec, tikz, harmony}
\title{\texttt{manimgl}常见问题}

\author{鹤翔万里}
\institute{\textsc{manim-kindergarten}}

\version{1.0}
\date{\zhtoday}

\lstset{
	basicstyle=\ttfamily
}

\begin{document}
	
\maketitle

\centerline{
	\includegraphics[height=3\baselineskip]{assets/Logo.png}
}

\newpage


\section*{一切之前}

\texttt{manim}是一个使用python制作视频的动画引擎。学习它你首先要会一点python,至少要学会python的基础语法、
模块的调用以及类的基础知识(如果想要阅读源码,还需要掌握更多python面向对象的知识)。没有python的知识
学\texttt{manim}是毫无意义的,会非常吃力,也会遇到非常多的问题。学会了python之后便可以少走非常多的弯路,
这也是我们的忠告。\quad 因此,加入我们
\textsc{Manim-Kindergarten}\footnote{QQ群:862671480}首先要会使用python,我们会在入群问
题\footnote{\url{https://b23.tv/KmAvsG}}中进行一个最基础的检测。

\textbf{\underline{最后,本常见问题文档基于\texttt{manim}的\texttt{master}分支版本}}\footnote{即新版\texttt{manim},原\texttt{shaders}分支版本}\textbf{\underline{,此文档中}}
\textbf{\underline{未出现的问题可能在旧版也会有同样问题,所以推荐两个常见问题文档}}\\
\textbf{\underline{一起查阅。}}

\subsection*{如何提问:}

\begin{enumerate}[I.]
	\item 在mk群里提问,首先需要阅读完本常见问题文档和基于旧版的常见问题文档
	
	\item 确保你问的问题不是由于python基础语法问题而造成的错误。
	
	\item 如果在\texttt{manim}使用过程中出现了报错:
	
	\begin{enumerate}[1.]
		\item 确保问题不在本文档和旧版常见问题文档中
		\item 将你的代码和\textbf{完整的}报错信息全部发送到群中,这样会方便群友为你解答
	\end{enumerate}

	\item 如果你想要实现某个效果,但不知道如何操作。请将你想要达到的效果详细易懂地描述出来。
	\item 语气友善,态度谦虚,避免造成不必要的纠纷。
\end{enumerate}

\newpage

\subsection*{教程推荐:}

\begin{enumerate}[I.]
	\item Python教程:
	
	\begin{enumerate}[1.]
		\item \href{https://www.ituring.com.cn/book/1861}{《Python编程:从入门到实践》},ISBN 978-7-115-42802-8
		
		\item \href{https://www.ituring.com.cn/book/1564}{《流畅的Python》},ISBN 978-7-115-45415-7(适合掌握Python基础知识后进阶)
		
		\item 菜鸟教程-Python3教程 \url{https://www.runoob.com/python3/python3-tutorial.html}
	\end{enumerate}

	\item \texttt{manim}教程:
	
	\begin{enumerate}[1.]
		\item \texttt{manim}教程文档(制作中):\url{https://manim.ml/}
		
		\item \texttt{MK}制作的系列视频教程(制作中)
		\begin{itemize}
			\item \url{https://space.bilibili.com/171431343/favlist?fid=947158443}
		\end{itemize}
	
		\item \texttt{MK}制作的视频源码(videos/)和常用自定义类(utils/)
	
		\item 群主\texttt{cigar666}的B站专栏
		\begin{itemize}
			\item \url{https://www.bilibili.com/read/readlist/rl82339}
		\end{itemize}
	
		\item \texttt{pdcxs}大大转载的\texttt{manim}教程
		\begin{itemize}
			\item \url{https://www.bilibili.com/video/av64023740}
			\item 源码 \url{https://github.com/Elteoremadebeethoven/AnimationsWithManim}
		\end{itemize}
	
		\item \texttt{GitHub}上\texttt{cai-hust}的中文教程
		\begin{itemize}
			\item \url{https://github.com/cai-hust/manim-tutorial-CN}
		\end{itemize}
	
		\item 看\texttt{manim}源码
	\end{enumerate}

	\item 新版\texttt{manim}教程:
	
	\begin{enumerate}[1.]
		\item 新版\texttt{manim}官方文档:\url{https://3b1b.github.io/manim/}
		
		\item 新版\texttt{manim}中文文档:\url{https://manim.ml/shaders/}
	\end{enumerate}

	\item OpenGL及shaders教程(Grant亲自推荐)
	\begin{enumerate}[1.]
		\item The Book of Shaders: \url{https://thebookofshaders.com/}
		
		\item Python \& OpenGL for Scientific Visualization: \url{https://www.labri.fr/perso/nrougier/python-opengl/}
		
		\item Geometry Shader: \url{https://learnopengl.com/Advanced-OpenGL/Geometry-Shader}
	\end{enumerate}
\end{enumerate}

\newpage

\tableofcontents

\newpage

\section{安装问题}

\begin{note}
	这里未提到的部分安装时会出现的问题请见旧版常见问题文档“manim常见问题v3.1”
\end{note}

\subsection{Python问题}

\subsubsection*{Q1: 关于\texttt{pip install manimlib}}
如果想要安装新版,请不要这样做。pypi上目前的\texttt{manimlib}仍为旧版,即目前\texttt{cairo-backend}分支中的版本。
安装新版目前只有克隆下GitHub上的3b1b/manim的master分支并直接使用\footnote{未来会上传到\texttt{manimgl}包中}

\subsubsection*{Q2: 无法安装包\texttt{mapbox\_earcut}}
如果在安装包时出现报错
\begin{lstlisting}[frame=none, columns=flexible]
CMake must be installed to build the following extensions
\end{lstlisting}

先通过运行\texttt{pip install cmake}安装\texttt{cmake},再尝试。

\newpage

\section{运行时问题}

\subsection{\texttt{import}问题}

\subsubsection*{Q1: 无法导入\texttt{OpenGL},抛出\texttt{ImportError}}
在最新的macOS Big Sur里,\texttt{OpenGL}的默认位置被从\texttt{library}改到了\texttt{library}下面的
\texttt{frameworks}中,所以在目前的版本中\texttt{PyOpenGL}找不到原来的位置,会导致报错
\begin{lstlisting}[frame=none, columns=flexible]
ImportError: (`Unable to load OpenGL library', `dlopen(OpenGL, 10): image not found', `OpenGL', None)
\end{lstlisting}

为此,长期的解决方法是等待接下来的Python更新。短期的解决方法是改变PyOpenGL的读取位置,步骤如下:
\begin{enumerate}[1.]
	\item 找到Python下\texttt{PyOpenGL}的文件\texttt{OpenGL/platform/ctypesloader.py},并且在做出任何改动之前做好完整的备份
	\item 找到其中的\texttt{fullName = util.find\_library( name )}
	\item 把它更改为\texttt{fullName = `/System/Library/Frameworks/OpenGL.fr}\\\texttt{amework/OpenGL'}
\end{enumerate}

\subsection{\LaTeX 问题}
具体\LaTeX 问题请见旧版常见问题文档“manim常见问题v3.1”

\subsubsection*{Q1: 如何使用中文}
如果要通过\LaTeX (即\texttt{manim}中的类\texttt{TexText})使用中文,需要在\texttt{custom\_defaults.yml}文件中进行更改。

将其中的\texttt{tex:}部分改为
\begin{lstlisting}[frame=none, columns=flexible]
tex:
  text_to_replace: "[tex_expression]"
  executable: "xelatex -no-pdf"
  template_file: "ctex_template.tex"
  intermediate_filetype: "xdv"
\end{lstlisting}

另外也请注意,\texttt{Text}和\texttt{TexText}不同,具体区别下面会提到。

\subsubsection*{Q2: 怎么检测少了哪个包}
尝试运行群文件“入群必看!!”中的\texttt{check\_latex\_packages.py}文件,其中哪个包检查后为\texttt{failed},则缺少了哪个包。
可以安装这个包,或者不必要的话将其从\texttt{ctex\_template.tex}中删除。

\subsection{dvisvgm问题}
\subsubsection*{Q1: 报错\texttt{OSError: C:\textbackslash Users\textbackslash ...\textbackslash Temp\textbackslash Tex\textbackslash <...>.svg not Found}}
如果你运行\texttt{manim}的位置不在C盘,那么请按照以下处理;如果你运行的位置在C盘,请看旧版常见问题文档中“dvisvgm问题”部分。

处理这个问题有两种方法:
\begin{enumerate}[1.]
	\item 更换2.8以上版本的dvisvgm
	\item 在\texttt{custom\_defaults.yml}中为\texttt{temporary\_storage}设置一个在同一磁盘下的路径
\end{enumerate}

\subsection{命令行选项问题}
\subsubsection*{Q1: \texttt{-pl}选项不能用了怎么办}
在新版中,命令行参数有了很大改变。\texttt{-p}选项已经删除,取而代之的是\texttt{-o}选项(在保存文件后打开文件)。
目前常用的选项如下\footnote{全部选项见文档: \url{https://manim.ml/shaders/getting\_started/configuration.html\#id4}}:
\begin{itemize}
	\item \texttt{-w} 将场景写入视频文件
	\item \texttt{-o} 将场景写入文件,并打开
	\item \texttt{-s} 直接跳到视频最后一帧,但是并不保存
	\item \texttt{-s} 与 \texttt{-o} 连用(\texttt{-s -o}或\texttt{-so}),保存并打开视频最后一帧
\end{itemize}

\subsubsection*{Q2: 怎么调整导出画质}
目前默认了四种画质
\begin{itemize}
	\item \texttt{low}: 480p15
	\item \texttt{medium}: 720p30
	\item \texttt{high}: 1080p30
	\item \texttt{ultra\_high}: 2160p60
\end{itemize}

可以通过更改\texttt{custom\_defaults.yml}中的\texttt{camera\_qualities.def}\\\texttt{ault\_quality}来设置默认质量。
也可以通过使用\texttt{-l,-m,-\!-hd,-\!-uhd}四个选项指定质量。也可以通过\texttt{-r WxH}指定分辨率、\texttt{-\!-frame\_rate FRAME\_RATE}
指定帧率。

\subsection{窗口问题}
\subsubsection*{Q1: 在运行的时候为什么会弹出一个窗口,如何交互}
这是新版本的特性,有实时播放的窗口,并且可以通过鼠标键盘进行交互。默认支持的交互操作有:
\begin{itemize}
	\item 滚动鼠标中键来上下移动画面
	\item 按住键盘上\texttt{z}键的同时滚动鼠标中键来缩放画面
	\item 按住键盘上\texttt{s}键的同时移动鼠标来平移画面
	\item 按住键盘上\texttt{d}键的同时移动鼠标来改变三维视角
\end{itemize}

需要注意:请确保你的输入法在英文模式,使用中文模式按上述按键无效。

\subsubsection*{Q2: \texttt{self.embed()}模式下窗口卡出,而且无法交互}
在进入\texttt{self.embed()}打开了iPython终端后,窗口暂时无法触碰,否则会出现无响应的情况(但是只要新的命令输入后就会恢复)。

要想在这是也通过鼠标进行交互,先输入\texttt{touch()}或者\texttt{self.interact()}。

\subsubsection*{Q3: 如何调整窗口的位置}
在\texttt{custom\_defaults.yml}中,更改\texttt{window\_position}。也可以使用\texttt{-f}选项来全屏显示窗口。

\subsection{文字问题}
\subsubsection*{Q1: 有些中文无法正常显示,出现实心等问题怎么办}
这是新版的巨大bug,目前还没有修复方法。如果大量使用文字的话,还是推荐使用旧版qwq

\subsubsection*{Q2: \texttt{Tex}、\texttt{TexText}、\texttt{Text}有什么区别}
旧版本中的\texttt{TexMobject}和\texttt{TextMobject}改了名字。\texttt{TexMobject}改为了\texttt{Tex},\texttt{TextMobject}
改为了\texttt{TexText}。
\begin{itemize}
	\item \textbf{\texttt{Tex}}: 使用\LaTeX 的公式环境,使用\LaTeX 语法
	\item \textbf{\texttt{TexText}}: 使用\LaTeX 的正文环境,使用\LaTeX 语法
	\item \textbf{\texttt{Text}}: 使用cairo来生成svg文字,不使用\LaTeX 语法,可以更换字体
\end{itemize}

\newpage

\section{其他问题}

\subsubsection*{Q1: 新旧版\texttt{manim}有什么区别}
\addcontentsline{toc}{subsection}{Q1: 新旧版\texttt{manim}有什么区别}

新版\texttt{manim}使用\texttt{OpenGL}和\texttt{moderngl}来进行GPU渲染,会有更快的速度,
也支持实时渲染和交互操作。更改了很多底层逻辑和结构,也调整了一些小的用法。
具体的所有用法上的更改待更新。

关于三个版本的\texttt{manim}的简要说明在\href{https://github.com/3b1b/manim/issues/1243}{\#1243}

\newpage

\section{注意}

如果有以上之外的问题,可以在群里提出,也可以在GitHub上提出issue,或者按照下图操作

    \begin{figure}[h]
        \begin{center}
            \includegraphics[width=6cm]{assets/grant.png}
        \end{center}
    \end{figure}

也请注意群规第 3,4 条
\begin{itemize}
    \item 3.虽为 manim 交流群,但不要一有问题就提出来,简单的问题能自己解决最好,不能解决时再寻求帮助
    \item 4.群主和管理员平时较忙,有时若不能及时回复敬请谅解
\end{itemize}

\begin{center}
\textbf{最后,祝大家好运(*^-^*)}
\end{center}

\newpage

\section{附:ChangeLog}

\subsection*{v1.0}
\begin{itemize}
	\item 开始为新版manim编写常见问题文档
\end{itemize}

\end{document}